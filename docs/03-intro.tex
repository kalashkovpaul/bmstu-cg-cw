\section*{ВВЕДЕНИЕ}
\addcontentsline{toc}{section}{ВВЕДЕНИЕ}

Моделирование твёрдых тел является неотъемлемой частью проектирования и разработки различных изделий, например, инженерных деталей. 
Все тела можно разделить на базовые и составные. 
К базовым относят примитивы: параллелепипед, цилиндр, шар, конус и др., в то время как составные можно сформировать в результате операций над базовыми (используя булевы функции сложения, пересечения и вычитания). 

\textbf{Целью работы}: проектирование программного обеспечения, позволяющего моделировать твердотельные модели на основе примитивов (куб, цилилндр, сфера, тор) и логических операций (объединение, пересечение, вычитание).
Для достижения поставленной цели необходимо выполнить следующие задачи:
\begin{enumerate}[label=\arabic*)]
	\item изучить методы представления твёрдых тел и выбрать из них наиболее подходящий;
	\item изучить методы рендера смоделированного тела и выбрать из них наиболее подходящий;
	\item изучить методы преобразования и визуализации объектов и выбрать из них наиболее подходящие;
	\item спроектировать программное обеспечение, соответствующее поставленной цели, с использованием выбранных методов представления, рендера, преобразования и визуализации объектов.
\end{enumerate}

