\section*{ВВЕДЕНИЕ}
\addcontentsline{toc}{section}{ВВЕДЕНИЕ}

TODO введение переработать, как в НИР - сухо и

Сегодня для повышения качества разрабатываемых продуктов, а также для увеличения эффективности труда, обычное двумерное проекционное черчение постепенно заменяется трёхмерным моделированием, которое\textbf{\textit{ работает с объектами, состоящими из замкнутого контура}}. TODO конец пред
Такой подход помогает обеспечить достаточно \textbf{\textit{полное описание}} трёхмерной геометрической формы.


Моделирование твёрдых тел является важной частью проектирования и разработки различных изделий, например, инженерных деталей. 
Все тела можно разделить на базовые и составные. 
К базовым относят примитивы: параллелепипед, шар, цилиндр, конус и др. 
\textit{\textbf{Тем не менее, в жизни можно редко встретить объекты}}, состоящие из одного базового тела, ведь обычно тела сложны по своей структуре и относятся к числу составных тел. 
Такие тела можно сформировать в результате операций над базовыми (используя булевы функции сложения, пересечения и вычитания). 
Существует несколько способов представления таких тел, из которых необходимо выбрать наиболее подходящий. TODO может не необходимо выбрать , а просто существуют
TODO это всё потом, после постановки задачи
После окончания моделирования тела наступает этап визуализации, в котором необходимо предусмотреть возможность просмотра модели с разных ракурсов (камер). 
Для этой цели также существует несколько способов, каждый из которых имеет свои преимущества и недостатки. 
Из них также необходимо выбрать оптимальный.

После создания тела его нужно отрисовать. 
Все приведённые выше действия наводят на мысль о создании специального программного обеспечения, которое объединит в себе решение озвученных задач и приведёт к конечному результату --- созданию твердотельной модели.

\textbf{Целью работы} является проектирование программного обеспечения, позволяющего моделировать твердотельные модели на основе примитивов и логических операций. 
Таким образом, необходимо выбрать оптимальные алгоритмы представления твердотельной модели, её преобразований \textit{\textbf{TODO почётче, каких преобразования}}, визуализации, а также программной обработки, спроектировать процесс моделирования и предоставить схему для его реализации. 
\pagebreak