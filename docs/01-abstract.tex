\section*{РЕФЕРАТ}

Расчетно-пояснительная записка \pageref{LastPage} с., \totalfigures\ рис., \totaltables\ табл., 16 ист.

КОМПЬЮТЕРНАЯ ГРАФИКА, КОНСТРУКТИВНАЯ БЛОЧНАЯ ГЕОМЕТРИЯ, МАРШИРОВАНИЕ ЛУЧЕЙ, ТВЕРДОТЕЛЬНЫЕ МОДЕЛИ

TODO всё в прошлом времени убрать

Цель работы --- проектирование программного обеспечения, позволяющего моделировать твердотельные модели на основе примитивов и логических операций. 

TODO переписать: Как метод создания модели, при помощи которой будет 
решаться  задача,  была  выбрана  конструктивная  блочная геометрия  (CSG), 
рендера  ---  Ray Marching.
Для  преобразования  модели  были  использованы 
матрицы преобразований, для придания им трёхмерного вида --- шейдеры.

Было разработано  программное обеспечение, предназначенное для создания моделей на основе объектов (куб, 
сфера, цилиндр и тор) с использованием операций пересечения, объединения и 
вычитания.  
Были  проанализированы  различные  алгоритмы,  методы  создания, 
преобразования  и  отображения  модели,  выбраны  наиболее  подходящие  для поставленной  задачи  технологии,  а  также  разработаны  алгоритмы  для  их 
программной реализации. 
Была разработана архитектура приложения, а также 
диаграммы используемых модулей. 

\pagebreak