\section*{РЕФЕРАТ}

Расчетно-пояснительная записка \pageref{LastPage} с., \totalfigures\ рис., \totaltables\ табл., 20 ист.

КОМПЬЮТЕРНАЯ ГРАФИКА, КОНСТРУКТИВНАЯ БЛОЧНАЯ ГЕОМЕТРИЯ, МАРШИРОВАНИЕ ЛУЧЕЙ, ТВЕРДОТЕЛЬНЫЕ МОДЕЛИ

Цель работы --- разработка программы, позволяющей моделировать твердотельные модели на основе примитивов и логических операций.

Методом представления моделия является  конструктивная  блочная геометрия  (CSG),
рендера  ---  Ray Marching.
Для  преобразования  модели  используются
матрицы преобразований, для придания им трёхмерного вида --- шейдеры.

Результаты: разработа программа, предназначенная для создания моделей на основе объектов (куб, цилиндр, сфера и тор) с использованием операций пересечения, объединения и
вычитания.
Проанализированы  различные  алгоритмы,  методы  представления,
преобразования  и  отображения  модели,  выбраны  наиболее  подходящие  для поставленной  задачи  технологии,  а  также  разработаны  алгоритмы  для  их
программной реализации.
Проведено исследование быстродействия программы при запуске на встроенной и дискретной видеокартах.

\pagebreak