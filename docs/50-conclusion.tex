\section*{ЗАКЛЮЧЕНИЕ}
В  ходе  выполнения  курсовой работы  было  разработано  программное обеспечение, предназначенное для создания моделей на основе объектов (куб, цилиндр, сфера, тор) с использованием операций логического пересечения, объединения и 
вычитания.  
Были  проанализированы  различные  алгоритмы,  методы  представления, 
преобразования  и  отображения  модели,  выбраны  наиболее  удоволетворяющие  поставленной  задачи  технологии,  а  также  разработаны  алгоритмы  для  их 
программной реализации. 
Разработанная программа позволяет пользователю создавать твердотельные модели на основе примитивов (куб, цилиндр, сфера и тор) с использованием логических операций (объединение, пересечение, вычитание). 
Присутствует возможность добавлять на сцену объекты, удалять их и преобразовывать, а также менять позицию камеры.
Проведено исследование быстродействия программы при запуске на встроенной и дискретной видеокартах. Из результатов следуюет, что дискретная видеокарта позволяет получить большее количечство кадров в секунду, является приемлемым для использования программы в режиме реалььного времени (более 30 кадров в секунду), в то время как встроенной видеокарты недостаточно для выполнения программы в режиме реального времени.

\pagebreak