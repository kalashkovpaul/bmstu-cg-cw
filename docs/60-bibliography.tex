\section*{СПИСОК ИСПОЛЬЗОВАННЫХ ИСТОЧНИКОВ}
\addcontentsline{toc}{section}{СПИСОК ИСПОЛЬЗОВАННЫХ ИСТОЧНИКОВ}

\begingroup
\renewcommand{\section}[2]{}

\begin{thebibliography}{}
	\bibitem{cloning}
	Желонин А. В. Методические рекомендации по дисциплине «Компьютерная графика». --- Ульяновск: УлГУ, 2019. --- 30 с.
	
	\bibitem{brep}
	Вельтмандер П. В. Машинная графика. Книга 2 Основные алгоритмы компьютерной графики. --- Новосибирск: Новосибирский государственный университет, 1997.
	
	\bibitem{numeric-octree}
	Витиска Н. И., Гуляев Н. А. Метод визуализации трёхмерных сцен и объектов воксельной графики для систем имитационного моделирования. --- Таганрог: Таганрогский институт им. А. П. Чехова Ростовского государственного экономического университета, 1962.
	
	\bibitem{sweeping}
	Новокщенов С. Л., Черных Д. М. Компьютерная графика: учебное пособие. --- Воронеж: ФГБОУ ВО
	«Воронежский государственный технический университет», 2017.
	
	\bibitem{csg}
	ГОСТ Р ИСО 10303-515-2007. Системы автоматизации производства и их интеграция. Представление данных об изделии и обмен этими данными. Часть 515. Прикладные интерпретированные конструкции. Конструктивная блочная геометрия
	
	\bibitem{rasterization}
	Белов Л. Б., Довгаль В. М., Гордиенко В. В. Растеризация точечных графических объектов на основе продукционной алгоритмической схемы. --- Курск: Курский государственный университет, ООО «Конус-Медик», 2012.
	
	\bibitem{raytracing}
	Шикин Е. В., Боресков А. В. Компьютерная графика. Динамика, реалистичные изображения. --- М: ДИАЛОГ-МИФИ, 1995.
	
	\bibitem{raycasting}
	Евстратов, В. В. Создание программы визуализации псевдотрехмерного изображения с помощью рейкастинга / В. В. Евстратов. — Текст : непосредственный // Молодой ученый. — 2020. — № 50 (340). — С. 12-15. — URL: https://moluch.ru/archive/340/76504/ (дата обращения: 01.11.2022).
	
	\bibitem{raymarching}
	Adrian Biagioli Raymarching Distance Fields: Concepts and Implementation in Unity. --- Pittsburgh: Carnegie Mellon University School of Computer Science, 2016.

	\bibitem{transformations}
	Коротаев А. И., Кузовлев В. И. Моделирование 3D объектов. --- Инженерный журнал: наука и инновации, 2013.
	
	\bibitem{shaders}
	Газизов В. Р. Программное профилирование работы графического процессора. --- Приволжский научный вестник, издательство Индивидуальный предприниматель Самохвалов Антон Витальевич (Ижевск), 2014.
	
	\bibitem{js}
	JavaScript --- official site [Электронный ресурс]. --- Режим доступа: https://www.javascript.com/ (дата обращения: 18.07.2022).
	
	\bibitem{threejs}
	Three JS [Электронный ресурс]. --- Режим доступа: https://threejs.org/docs/ (дата обращения: 18.07.2022).
	
	\bibitem{stats}
	JavaScript Performance Monitor [Электронный ресурс]. --- Режим доступа: https://github.com/mrdoob/stats.js (дата обращения: 18.07.2022).
	
	\bibitem{datgui}
	dat.GUI. A lightweight graphical user interface for changing variables in JavaScript [Электронный ресурс]. --- Режим доступа: https://github.com/dataarts/dat.gui (дата обращения: 18.07.2022).
	
	\bibitem{vscode}
	Visual Studio Code --- Code Editing. Redefined [Электронный ресурс]. --- Режим доступа: https://code.visualstudio.com (дата обращения: 18.07.2022).
\end{thebibliography}
\endgroup

\pagebreak